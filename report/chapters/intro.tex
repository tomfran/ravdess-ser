\section{Introduction}

Speech emotion recognition is one of the many tasks that machine learning 
can tackle on audio signals. The requirement is, staring from raw audio files, 
to assign them an emotion, like neutral, happy or sad. 
As with many others classification tasks, one can follow a more classical 
approach, extracting features from raw files to capture important audio characteristics
and to build a classifier on top of that. 

Other methods are possibile, like using Convolutional neural networks on raw data, 
or deep learning networks with an high number of layers.

In contrast with other fields of work, audio with labelled emotions datasets are not
really big, this can be due to the fact that recording, preprocessing and validating 
emotions can be really expensive and time consuming, but it clearly creates an additional 
difficulty when trying to classify them. For this purpose, data augmentation takes a big role 
in this study as it can, if done right, mitigate the problems of data scarcity. 
